\documentclass{beamer}
\mode<presentation>
{
\usetheme{Madrid}
\usecolortheme{seagull}
}

\usepackage{graphicx}
\usepackage{amsmath}
\usepackage{amssymb}

\title[Docker Workshop]{Docker Workshop} 

\author{Jaime Canizales} 
\institute[Hunter College] 
{
City University of New York \\ 
\medskip
\textit{jaime.canizales@hunter.cuny.edu} 
}
\date{\today} 
\begin{document}


\begin{frame}
\titlepage 
\end{frame}


\begin{frame} \frametitle{Overview} 
\tableofcontents
\end{frame}


\section{Introduction}
\begin{frame}\frametitle{What is Docker?}
\begin{block}{Problem Statement}
Docker is a platform for creating, running, and managing containers—lightweight,
portable environments that package an application and its dependencies. 
Containers ensure the app runs consistently across different systems. Unlike 
virtual machines, containers share the host OS, making them faster and more 
efficient. Docker simplifies development, testing, and deployment, with tools
like Docker Hub for sharing images and orchestration support for scaling.
\end{block}
\end{frame}


\begin{frame}\frametitle{Why use Docker?}
\begin{itemize}
\item \textbf{Consistency:} Ensures that applications run the same way regardless of the host environment (development, testing, production).
\item \textbf{Portability:} Docker containers can run on any system with Docker installed, including on-premises, cloud, or hybrid setups.
\item \textbf{Isolation:} Containers encapsulate applications and their dependencies, avoiding conflicts between different environments or applications.
\item \textbf{Efficiency:} Lightweight compared to virtual machines (VMs), as they share the host OS kernel and require fewer resources.
\item \textbf{Scalability:} Makes it easy to scale applications horizontally by running multiple container instances.
\end{itemize}
\end{frame}


\begin{frame}\frametitle{Why use Docker?(Cont.)}
\begin{itemize}
\item \textbf{Faster Development and Deployment:} Speeds up workflows with instant container creation and simplified testing across environments.
\item \textbf{Microservices Architecture:} Ideal for breaking down applications into smaller, manageable, and independently deployable services.
\item \textbf{Reproducibility:} Developers can create a predictable environment using Dockerfiles and Docker Compose, reducing "it works on my machine" issues.
\item \textbf{DevOps Integration:} Fits seamlessly into CI/CD pipelines for automated testing, integration, and deployment.
\item \textbf{Community and Ecosystem:} Extensive support with prebuilt images available on Docker Hub and integration with tools like Kubernetes for orchestration.
\end{itemize}
\end{frame}


\section{Technical Information}
\begin{frame}\frametitle{Installing Docker}
\begin{itemize}
\item \href{https://docs.docker.com/get-started/get-docker/}{Docker install link}   
\end{itemize}
\end{frame}


\begin{frame}\frametitle{Docker Technical Terminology}
\begin{itemize}
\item \textbf{Docker Client(CLI)}: The command-line interface used by users
to interact with the Docker Daemon.
\item \textbf{Docker Daemon}: The core background process that manages Docker
objects like containers, images, networks, and volumes. It listens for API
requests and communicates with the CLI and other components.
\item \textbf{Docker Engine}: Refers to the system containing both Docker CLI and Docker Daemon
\item \textbf{Docker Desktop}: Refers to the system containing both Docker Engine and a Docker graphical user interface(gui)
\item \textbf{Docker Registry}: is a centralized storage and distribution system for Docker images.(version control for docker images) (we will use Dockerhub)
\end{itemize}
\end{frame}


\begin{frame}\frametitle{Docker Container}
\begin{block}{Docker Container}
A Docker Container is a lightweight, portable runtime environment created from 
a Docker image. It runs an application and its dependencies in isolation, 
sharing the host system's kernel for efficiency.
\end{block}
\end{frame}


\begin{frame}\frametitle{Docker Image}
\begin{block}{Docker Image}
A Docker Image is a lightweight, immutable blueprint for creating containers. 
It includes everything needed to run an application: code, dependencies, 
libraries, and settings. Images are built in layers, stored in registries,
and used to launch containers.(show example of a dockerfile)
\end{block}
\end{frame}


\begin{frame}\frametitle{Dockerfile}
\begin{block}{Dockerfile}
A Dockerfile is a text file that contains instructions to build a Docker image.
Each line in the Dockerfile specifies a step in the build process.(Go over image)
\end{block}
\end{frame}


\begin{frame}\frametitle{Basic Docker Commands For Start Up}
\begin{itemize}
\item docker build -t image-name . (creates an image named image-name from a dockerfile inside the cwd)
\item docker run --name test-container -it --rm image-name (creates a containe named test-container from an image)  
\item docker ps -a (Shows container ids)
\item docker exec -it container-id bash (opens another terminal in a current container)
\end{itemize}
\end{frame}
    

\begin{frame}\frametitle{Basic Docker Commands for Clean Up}
\begin{itemize}
\item docker rm container-id (deletes a single container)
\item docker container prune --force (deletes all unused containers)
\item docker image prune --force (deletes all unused or dangling images)
% \item docker volume prune --force (deletes all unused or dangling volumes)
% \item docker network prune --force
\end{itemize}
\end{frame}


\begin{frame}\frametitle{Dev Containers}
\begin{itemize}
\item Docker Dev Containers are containerized development environments created 
using Docker. They package tools, dependencies, and configurations, providing
consistent, isolated setups for coding across machines. Often used with tools
like VS Code for seamless integration.
\item Basically you can use vscode and a json file to specify configuration and run containers 
without having to use the command line
\item You can even include remote images in devcontainers.json
by just adding one line. e.g. "image" : "matimoreyra/opencv:latest", (image with opencv for c++)
\end{itemize}
\end{frame}


\begin{frame}\frametitle{DockerHub}
\begin{block}{DockerHub}
Docker Hub is a cloud-based repository for building, sharing, and managing Docker images. 
Instead of storing and building image locally, you can build them on the cloud instead!
Or you can create contianers from pre-existing images on DockerHub! DockerHub is one of the 
many examples of a docker registry(a very popular one). 
\end{block}
\end{frame}


\begin{frame}\frametitle{DockerHub Commands}
\begin{itemize}
\item docker commit container-id repository-name:tag (to save changes on container to remote image)
\item docker tag old-repository-name:tag dockerhub-username/new-repository-name:tag (add a tag to an image and/or give an image an alias)
\item docker push dockerhub-username/repository-name:tag (update the remote copy of the image on the cloud)
\item docker pull repository-name:tag (make a local copy of a remote image from the cloud)
\end{itemize}
\end{frame}


\section{Conclusion}
\begin{frame}\frametitle{Check out the Docker Examples provided in this repo}
\begin{itemize}
\item authentication for node and psql image
\item opencv image
\item python stuff added to image
\item Extension name: latex workshop james yu, then check out file :./.vscode/settings.json to see how to include the docker image for latex
\end{itemize}
\end{frame}


% \begin{frame}\frametitle{Software Architecture}
% \begin{figure}
% \includegraphics[width=12cm]{software_architecture.png}
% \begin{itemize}
% \item Ensure everything is in the same vpc(virtual private network), like in picture.
% \end{itemize}
% \item Connect vscode Documentation: 
% \end{figure}  
% \end{frame}

\end{document}